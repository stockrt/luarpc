% DOC CLASS
\documentclass[11pt]{article}

% ENCODING
\usepackage[utf8]{inputenc}
\usepackage[brazilian]{babel}

% OUTPUT
\usepackage[nottoc]{tocbibind}
\usepackage{hyperref}
\usepackage{fullpage}
\usepackage{graphicx}
\usepackage{multicol}
\usepackage{makeidx}
\usepackage{float}
\usepackage{array}
\usepackage{boxedminipage}
\usepackage{amsmath}
\usepackage{enumitem}
\usepackage{amsfonts}
\usepackage{setspace}
\usepackage{comment}
\usepackage[toc,xindy]{glossaries}

% SOURCE CODE
\usepackage{listings}
\usepackage{color}
\usepackage{xcolor}
\usepackage{caption}
\DeclareCaptionFont{white}{\color{white}}
\DeclareCaptionFormat{listing}{\colorbox{gray}{\parbox{\textwidth}{#1#2#3}}}
\captionsetup[lstlisting]{format=listing,labelfont=white,textfont=white}

% MARGIN
\setlength{\oddsidemargin}{0.5cm} \setlength{\textwidth}{15cm}
\setlength{\topmargin}{-1.5cm} \setlength{\textheight}{22.3cm}%{24.7cm}
\setlength\parindent{0pt} % Removes all indentation from paragraphs
\renewcommand{\labelenumi}{\alph{enumi}.} % Make numbering in the enumerate environment by letter rather than number (e.g. section 6)

% GLOSSARIES
\makeglossaries
\newacronym{rpc}{RPC}{Remote Procedure Calls}

% DOCUMENT
\begin{document}

\title{Disciplina de Sistemas Distrbuídos [INF 2545] \\ Biblioteca RPC -
Trabalho 1}
\author{Rogério Carvalho Schneider}
\date{Abril 2014}
\maketitle

\begin{center}
\begin{tabular}{l r}
Data dos testes: & 14 de Abril de 2014 \\
Professora: & Noemi Rodriguez
\end{tabular}
\end{center}

\begin{abstract}
Relatório da implementação e testes realizados na biblioteca \textit{luarpc}.
\end{abstract}

\doublespacing

\section{Introdução}\label{sec:introduction}

Desenvolvemos uma biblioteca para auxílio a chamadas remotas de
procedimentos, \gls{rpc}, com o objetivo de enfrentar e resolver os problemas
de comunicação de rede em sistemas distribuídos. A programação de servidor e
cliente \gls{rpc} também serviu de exercício na definição, entendimento e
implementação de um protocolo de comunicação. Uma lista de discussão por
\textit{e-mail} for arranjada de forma que as propostas de protocolo fossem
comentadas pelos participantes do consórcio que patrocinaria o desenvolvimento
da biblioteca. Uma \gls{rfc} foi definida na lista de discussão e após algumas
rodadas de interação o consórcio chegou a um entendimento quanto ao protocolo a
ser implementado.

A biblioteca consiste no arquivo \textit{luarpc.lua}, que depende da
definição de um arquivo de interface, normalmente chamado de
\textit{interface.lua}. A biblioteca deve ser usada por diferentes
implementações de servidor e cliente, para tanto, um contrato foi firmado entre
os desenvolvedores (os alunos da disciplina) de maneira a seguirem uma \gls{api}
comum a todas as implementações da biblioteca \textit{luarpc}. Da mesma forma,
foi necessário entrar em acordo quanto ao protocolo de comunicação de rede e de
codificação/decodificação a ser utilizado. A implementação desta
biblioteca \gls{rpc} segue as linhas gerais definidas no enunciado do trabalho
\cite{trab1}.

A \gls{api} utilizada foi aquela do enunciado do trabalho, com os métodos
\textit{luarpc.createServant}, \textit{luarpc.waitIncoming} e
\textit{luarpc.createProxy} com exatamente o mesmo número e tipo de parâmetros
sugeridos pelo enunciado original. O protocolo sofreu pequenas alterações,
principalmente no que diz respeito a forma de codificação e decodificação do
conteúdo das mensagens enviadas. Particularmente, a regra definida para o
protocolo foi a seguinte:

\begin{itemize}
\item
Nomes de métodos chamados pelo cliente são enviados sem delimitadores;
\item
Parâmetros string são enviados com delimitadores "" envolvendo o texto
transmitido;
\item
Parâmetros void são transformados na nil e são enviados com delimitadores "" envolvendo o texto
transmitido. Resulta no envio de "nil";
\item
\end{itemize}

Ao se transmitir o conteúdo é necessário serializar o mesmo, codificando
conforme o seguinte acordo, no transmissor:
\begin{itemize}
\item
Usar sequência de escape para \, transformando em \\;
\item
Usar sequência de escape para \n, transformando em \\n;
\item
Usar sequência de escape para ", transformando em \";
\item
O processo inverso, de decodificação, deve ser feito no receptor.
\item
Os tipos válidos, \textit{char}, \textit{string}, \textit{double} e
\textit{void} tem o seguinte protocolo de
codificação/decodificação:
\item
\textit{char}: restrito a um caracter
codificação no transmissor: envolto em ""
exemplo: "a"
decodificação no receptor: remove ""
exemplo: a
\item
\textit{string}: sem restrição
codificação no transmissor: envolto em ""
exemplo: "abc"
decodificação no receptor: remove ""
exemplo: abc
\item
\textit{double}: apenas números inteiros ou de ponto flutuante
codificação no transmissor: não há
exemplo: 3.1415
decodificação no receptor: não há
exemplo: 3.1415
\item
\textit{void}: transformado em nil, envolto em ""
codificação no transmissor: nil e ""
exemplo: "nil"
decodificação no receptor: remove ""
exemplo: nil
\item
Todos os tipos devem ser enviados em apenas uma chamada de \textit{socket:send}
e recebidos em apenas uma chamada \textit{socket:receive}, desta maneira, o
correto escape de \textit{string} com quebras de linha (\n) é muito importante.

Ao perceber um erro no lado do cliente ou do servidor, o protocolo deve enviar
uma mensagem iniciada por "___ERRORPC:" contendo, possivelmente, a causa do
erro.
\end{itemize}

\section{Gargalos identificados}\label{sec:bottle}

\section{Melhorias futuras}\label{sec:future}

Diferença de tempo de entre chamadas sem persistência e com persistência de
conexão.

Cliente com loop de chamadas para alguns métodos especiais de teste, com maior
e menor overhead de rede e de encoding.

Servidor com pool de três conexões.
Como se comporta com apenas um cliente?
Como se comporta com três clientes?
Como se comporta com dez clientes?
Como se comporta cada caso com diferentes servidores, o que mantém e o que não
mantém pool de conexões?


 1 cliente - persist
01:46:24
foo took 2 seconds for server 5001 for 10000 runs
oid took 2 seconds for server 5001 for 10000 runs
min took 2 seconds for server 5001 for 10000 runs
min 10KB took 17 seconds for server 5001 for 10000 runs
men took 3 seconds for server 5001 for 10000 runs
tbl took 8 seconds for server 5001 for 10000 runs
foo took 3 seconds for server 5002 for 10000 runs
oid took 2 seconds for server 5002 for 10000 runs
min took 2 seconds for server 5002 for 10000 runs
min 10KB took 20 seconds for server 5002 for 10000 runs
men took 2 seconds for server 5002 for 10000 runs
tbl took 9 seconds for server 5002 for 10000 runs

01:47:39
foo took 2 seconds for server 5001 for 10000 runs
oid took 2 seconds for server 5001 for 10000 runs
min took 2 seconds for server 5001 for 10000 runs
min 10KB took 20 seconds for server 5001 for 10000 runs
men took 2 seconds for server 5001 for 10000 runs
tbl took 8 seconds for server 5001 for 10000 runs
foo took 3 seconds for server 5002 for 10000 runs
oid took 2 seconds for server 5002 for 10000 runs
min took 2 seconds for server 5002 for 10000 runs
min 10KB took 20 seconds for server 5002 for 10000 runs
men took 2 seconds for server 5002 for 10000 runs
tbl took 9 seconds for server 5002 for 10000 runs


00:04:18
foo took 11 seconds for server 5001 for 50000 runs
oid took 7 seconds for server 5001 for 50000 runs
min took 8 seconds for server 5001 for 50000 runs
min 10KB took 110 seconds for server 5001 for 50000 runs
men took 7 seconds for server 5001 for 50000 runs
tbl took 32 seconds for server 5001 for 50000 runs
foo took 10 seconds for server 5002 for 50000 runs
oid took 8 seconds for server 5002 for 50000 runs
min took 9 seconds for server 5002 for 50000 runs
min 10KB took 112 seconds for server 5002 for 50000 runs
men took 8 seconds for server 5002 for 50000 runs
tbl took 40 seconds for server 5002 for 50000 runs

00:44:11
foo took 13 seconds for server 5001 for 50000 runs
oid took 11 seconds for server 5001 for 50000 runs
min took 11 seconds for server 5001 for 50000 runs
min 10KB took 107 seconds for server 5001 for 50000 runs
men took 11 seconds for server 5001 for 50000 runs
tbl took 39 seconds for server 5001 for 50000 runs
foo took 14 seconds for server 5002 for 50000 runs
oid took 10 seconds for server 5002 for 50000 runs
min took 11 seconds for server 5002 for 50000 runs
min 10KB took 106 seconds for server 5002 for 50000 runs
men took 9 seconds for server 5002 for 50000 runs
tbl took 41 seconds for server 5002 for 50000 runs


 3 clientes - persist
01:50:06
foo took 4 seconds for server 5001 for 10000 runs
oid took 3 seconds for server 5001 for 10000 runs
min took 3 seconds for server 5001 for 10000 runs
min 10KB took 22 seconds for server 5001 for 10000 runs
men took 23 seconds for server 5001 for 10000 runs
tbl took 12 seconds for server 5001 for 10000 runs
foo took 10 seconds for server 5002 for 10000 runs
oid took 5 seconds for server 5002 for 10000 runs
min took 5 seconds for server 5002 for 10000 runs
min 10KB took 27 seconds for server 5002 for 10000 runs
men took 18 seconds for server 5002 for 10000 runs
tbl took 19 seconds for server 5002 for 10000 runs

01:50:35
foo took 6 seconds for server 5001 for 10000 runs
oid took 7 seconds for server 5001 for 10000 runs
min took 13 seconds for server 5001 for 10000 runs
min 10KB took 28 seconds for server 5001 for 10000 runs
men took 10 seconds for server 5001 for 10000 runs
tbl took 12 seconds for server 5001 for 10000 runs
foo took 7 seconds for server 5002 for 10000 runs
oid took 9 seconds for server 5002 for 10000 runs
min took 15 seconds for server 5002 for 10000 runs
min 10KB took 31 seconds for server 5002 for 10000 runs
men took 14 seconds for server 5002 for 10000 runs
tbl took 15 seconds for server 5002 for 10000 runs

01:50:36
foo took 7 seconds for server 5001 for 10000 runs
oid took 8 seconds for server 5001 for 10000 runs
min took 18 seconds for server 5001 for 10000 runs
min 10KB took 27 seconds for server 5001 for 10000 runs
men took 9 seconds for server 5001 for 10000 runs
tbl took 11 seconds for server 5001 for 10000 runs
foo took 8 seconds for server 5002 for 10000 runs
oid took 15 seconds for server 5002 for 10000 runs
min took 17 seconds for server 5002 for 10000 runs
min 10KB took 31 seconds for server 5002 for 10000 runs
men took 5 seconds for server 5002 for 10000 runs
tbl took 14 seconds for server 5002 for 10000 runs

01:54:44
foo took 7 seconds for server 5001 for 10000 runs
oid took 10 seconds for server 5001 for 10000 runs
min took 21 seconds for server 5001 for 10000 runs
min 10KB took 31 seconds for server 5001 for 10000 runs
men took 10 seconds for server 5001 for 10000 runs
tbl took 16 seconds for server 5001 for 10000 runs
foo took 9 seconds for server 5002 for 10000 runs
oid took 17 seconds for server 5002 for 10000 runs
min took 18 seconds for server 5002 for 10000 runs
min 10KB took 33 seconds for server 5002 for 10000 runs
men took 6 seconds for server 5002 for 10000 runs
tbl took 12 seconds for server 5002 for 10000 runs

01:54:45
foo took 7 seconds for server 5001 for 10000 runs
oid took 7 seconds for server 5001 for 10000 runs
min took 16 seconds for server 5001 for 10000 runs
min 10KB took 30 seconds for server 5001 for 10000 runs
men took 13 seconds for server 5001 for 10000 runs
tbl took 15 seconds for server 5001 for 10000 runs
foo took 9 seconds for server 5002 for 10000 runs
oid took 7 seconds for server 5002 for 10000 runs
min took 16 seconds for server 5002 for 10000 runs
min 10KB took 29 seconds for server 5002 for 10000 runs
men took 15 seconds for server 5002 for 10000 runs
tbl took 16 seconds for server 5002 for 10000 runs

01:54:46
foo took 4 seconds for server 5001 for 10000 runs
oid took 5 seconds for server 5001 for 10000 runs
min took 4 seconds for server 5001 for 10000 runs
min 10KB took 26 seconds for server 5001 for 10000 runs
men took 25 seconds for server 5001 for 10000 runs
tbl took 17 seconds for server 5001 for 10000 runs
foo took 11 seconds for server 5002 for 10000 runs
oid took 7 seconds for server 5002 for 10000 runs
min took 6 seconds for server 5002 for 10000 runs
min 10KB took 25 seconds for server 5002 for 10000 runs
men took 16 seconds for server 5002 for 10000 runs
tbl took 20 seconds for server 5002 for 10000 runs


00:30:35
foo took 16 seconds for server 5001 for 50000 runs
oid took 16 seconds for server 5001 for 50000 runs
min took 15 seconds for server 5001 for 50000 runs
min 10KB took 88 seconds for server 5001 for 50000 runs
men took 98 seconds for server 5001 for 50000 runs
tbl took 82 seconds for server 5001 for 50000 runs
foo took 43 seconds for server 5002 for 50000 runs
oid took 45 seconds for server 5002 for 50000 runs
min took 19 seconds for server 5002 for 50000 runs
min 10KB took 101 seconds for server 5002 for 50000 runs
men took 17 seconds for server 5002 for 50000 runs
tbl took 79 seconds for server 5002 for 50000 runs

00:30:35
foo took 30 seconds for server 5001 for 50000 runs
oid took 38 seconds for server 5001 for 50000 runs
min took 72 seconds for server 5001 for 50000 runs
min 10KB took 147 seconds for server 5001 for 50000 runs
men took 39 seconds for server 5001 for 50000 runs
tbl took 59 seconds for server 5001 for 50000 runs
foo took 38 seconds for server 5002 for 50000 runs
oid took 91 seconds for server 5002 for 50000 runs
min took 39 seconds for server 5002 for 50000 runs
min 10KB took 140 seconds for server 5002 for 50000 runs
men took 33 seconds for server 5002 for 50000 runs
tbl took 59 seconds for server 5002 for 50000 runs

00:30:34
foo took 31 seconds for server 5001 for 50000 runs
oid took 43 seconds for server 5001 for 50000 runs
min took 73 seconds for server 5001 for 50000 runs
min 10KB took 150 seconds for server 5001 for 50000 runs
men took 35 seconds for server 5001 for 50000 runs
tbl took 65 seconds for server 5001 for 50000 runs
foo took 57 seconds for server 5002 for 50000 runs
oid took 75 seconds for server 5002 for 50000 runs
min took 57 seconds for server 5002 for 50000 runs
min 10KB took 135 seconds for server 5002 for 50000 runs
men took 23 seconds for server 5002 for 50000 runs
tbl took 60 seconds for server 5002 for 50000 runs


 10 clientes - persist
02:04:46
foo took 24 seconds for server 5001 for 10000 runs
oid took 21 seconds for server 5001 for 10000 runs
min took 45 seconds for server 5001 for 10000 runs
min 10KB took 113 seconds for server 5001 for 10000 runs
men took 30 seconds for server 5001 for 10000 runs
tbl took 57 seconds for server 5001 for 10000 runs
foo took 26 seconds for server 5002 for 10000 runs
oid took 25 seconds for server 5002 for 10000 runs
min took 48 seconds for server 5002 for 10000 runs
min 10KB took 110 seconds for server 5002 for 10000 runs
men took 33 seconds for server 5002 for 10000 runs
tbl took 56 seconds for server 5002 for 10000 runs


 Table apenas
tblonly serialize took 4 seconds for 10000 runs
tblonly deserialize took 2 seconds for 10000 runs
tblonly serialize/deserialize took 6 seconds for 10000 runs

tblonly serialize took 22 seconds for 50000 runs
tblonly deserialize took 7 seconds for 50000 runs
tblonly serialize/deserialize took 32 seconds for 50000 runs


 1 cliente - close connection
02:23:45
foo took 4 seconds for server 5001 for 10000 runs
oid took 4 seconds for server 5001 for 10000 runs
min took 4 seconds for server 5001 for 10000 runs
min 10KB took 21 seconds for server 5001 for 10000 runs
men took 4 seconds for server 5001 for 10000 runs
tbl took 10 seconds for server 5001 for 10000 runs
foo took 5 seconds for server 5002 for 10000 runs
oid took 4 seconds for server 5002 for 10000 runs
min took 4 seconds for server 5002 for 10000 runs
min 10KB took 21 seconds for server 5002 for 10000 runs
men took 4 seconds for server 5002 for 10000 runs
tbl took 11 seconds for server 5002 for 10000 runs


 3 clientes - close connection
02:33:17
foo took 5 seconds for server 5001 for 10000 runs
oid took 7 seconds for server 5001 for 10000 runs
min took 8 seconds for server 5001 for 10000 runs
min 10KB took 32 seconds for server 5001 for 10000 runs
men took 10 seconds for server 5001 for 10000 runs
tbl took 17 seconds for server 5001 for 10000 runs
foo took 10 seconds for server 5002 for 10000 runs
oid took 9 seconds for server 5002 for 10000 runs
min took 8 seconds for server 5002 for 10000 runs
min 10KB took 35 seconds for server 5002 for 10000 runs
men took 6 seconds for server 5002 for 10000 runs
tbl took 19 seconds for server 5002 for 10000 runs

% BIB
\bibliographystyle{abnt-puc-rio}
\bibliography{reference}

\end{document}
